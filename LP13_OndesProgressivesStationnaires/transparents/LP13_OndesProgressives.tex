% IMPORTANT: PLEASE USE XeLaTeX FOR TYPESETTING
\documentclass[10pt]{beamer}
\input{pckg_beamer.tex}
\usepackage{tikz}
\usepackage{array}
\usepackage[scientific-notation=true]{siunitx}
\usetikzlibrary{matrix}
\newcommand{\diff}{\mathrm{d}}

\title{Leçon :Ondes progressives et stationnaires}

% document body
\begin{document}
\begin{frame}{}
    \titlepage

    \begin{tabularx}{\textwidth}{l@{:\,\,}X}
        \textbf{Niveau} 	  & CPGE PSI\\
        \textbf{Prérequis} & Mécanique de première année \\
        & Résolution d'équations différentielles \\
        & Phénomènes ondulatoires, vocabulaire longueur d'onde...
    \end{tabularx}
\end{frame}

\begin{frame}
    \tableofcontents
\end{frame}

\section{Propagation des ondes}
\subsection{Definition}
\subsection{Corde vibrante}
\subsection{Mise en équations}
\subsection{Généralisation}
\begin{frame}{\insertsubsection}
    Équation de Maxwell dans le vide: 
    \begin{equation}
        \begin{array}{cc}
            \vec{\nabla}\cdot \vec{E} = \dfrac{\rho}{\epsilon_0} & \vec{\nabla}\wedge \vec{E} = -\dfrac{\partial \vec{B}}{\partial t}\\
            \vec{\nabla}\cdot \vec{B} =0 &   \vec{\nabla}\wedge \vec{E} = \epsilon_0\mu_0\dfrac{\partial \vec{E}}{\partial t}
        \end{array}
    \end{equation}

    $$\vec{\nabla} \wedge \vec{\nabla} \vec{X} = \vec{\nabla}\left(\vec{\nabla}\cdot \vec{X}\right) - \Delta X$$
\end{frame}
\section{Solution générale de l'équation de d'Alembert: ondes progressives}
\subsection{Ondes progressives}
\subsection{Ondes progressives harmoniques}
\subsection{Relation de dispersion}
\section{Une autre famille d'ondes: ondes stationnaires}
\subsection{Solution de l'équation de d'Alembert}
\subsection{Corde de Melde}
\subsection{Onde stationnaire ou onde propagative}
\begin{frame}
    Merci pour votre attention
\end{frame}
\end{document}