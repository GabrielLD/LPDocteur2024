%!TEX encoding = UTF-8 Unicode
\documentclass[french, a4paper, 10pt, twocolumn, landscape]{article}



%% Langue et compilation

\usepackage[utf8]{inputenc}
\usepackage[T1]{fontenc}
\usepackage[french]{babel}
\usepackage{lmodern}       % permet d'avoir certains "fonts" de bonne qualite
\renewcommand{\familydefault}{\sfdefault}
%% LISTE DES PACKAGES

\usepackage{mathtools}     % package de base pour les maths
\usepackage{amsmath}       % mathematical type-setting
\usepackage{amssymb}       % symbols speciaux pour les maths
\usepackage{textcomp}      % symboles speciaux pour el text
\usepackage{gensymb}       % commandes generiques \degree etc...
\usepackage{tikz}          % package graphique
\usepackage{wrapfig}       % pour entourer a cote d'une figure
\usepackage{color}         % package des couleurs
\usepackage{xcolor}        % autre package pour les couleurs
\usepackage{pgfplots}      % pacakge pour creer des graph
\usepackage{epsfig}        % permet d'inclure des graph en .eps
\usepackage{graphicx}      % arguments dans includegraphics
\usepackage{pdfpages}      % permet d'insérer des pages pdf dans le document
\usepackage{subfig}        % permet de creer des sous-figure
% \usepackage{pst-all}       % utile pour certaines figures en pstricks
\usepackage{lipsum}        % package qui permet de faire des essais
\usepackage{array}         % permet de faire des tableaux
\usepackage{multicol}      % plusieurs colonnes sur une page
\usepackage{enumitem}      % pro­vides user con­trol: enumerate, itemize and description
\usepackage{hyperref}      % permet de creer des hyperliens dans le document
\usepackage{lscape}        % permet de mettre une page en mode paysage

\usepackage{fancyhdr}      % Permet de mettre des informations en hau et en bas de page      
\usepackage[framemethod=tikz]{mdframed} % breakable frames and coloured boxes
\usepackage[top=1.8cm, bottom=1.8cm, left=1.5cm, right=1.5cm]{geometry} % donne les marges
\usepackage[font=normalsize, labelfont=bf,labelsep=endash, figurename=Figure]{caption} % permet de changer les legendes des figures
\setlength{\parskip}{0pt}%
\setlength{\parindent}{18pt}
\usepackage{lewis}
\usepackage{bohr}
\usepackage{chemfig}
\usepackage{chemist}
\usepackage{tabularx}
\usepackage{pgf-spectra} % permet de tracer des spectres lumineux des atomes et des ions
\usepackage{pgf}

\usepackage{flexisym}
\usepackage{soul}
\usepackage{ulem}
\usepackage{cancel}

\usepackage{import}
\usepackage{physics}
\usepackage[outline]{contour} % glow around text
\tikzset{every shadow/.style={opacity=1}}


%% LIBRAIRIES

\usetikzlibrary{plotmarks} % librairie pour les graphes
\usetikzlibrary{patterns}  % necessaire pour certaines choses predefinies sur tikz
\usetikzlibrary{shadows}   % ombres des encadres
\usetikzlibrary{backgrounds} % arriere plan des encadres


%% MISE EN PAGE

\pagestyle{fancy}     % Défini le style de la page

\renewcommand{\headrulewidth}{0pt}      % largeur du trait en haut de la page
\fancyhead[L]{\textbf{\textcolor{cyan}{Cours}} - Thème 4 - La Terre un astre singulier}         % info coin haut gauche
\fancyhead[R]{\textit{Première Enseignement Scientifique}}  % info coin haut droit

% % bas de la page
% \renewcommand{\footrulewidth}{0pt}      % largeur du trait en bas de la page
% \fancyfoot[L]{}  % info coin bas gauche
\fancyfoot[R]{Lycée GT Jean Guéhenno}                         % info coin bas droit


\setlength{\columnseprule}{1pt} 
\setlength{\columnsep}{30pt}



%% NOUVELLES COMMANDES 

\DeclareMathOperator{\e}{e} % permet d'ecrire l'exponentielle usuellement


\newcommand{\gap}{\vspace{0.15cm}}   % defini une commande pour sauter des lignes
\renewcommand{\vec}{\overrightarrow} % permet d'avoir une fleche qui recouvre tout le vecteur
\newcommand{\bi}{\begin{itemize}}    % begin itemize
\newcommand{\ei}{\end{itemize}}      % end itemize
\newcommand{\bc}{\begin{center}}     % begin center
\newcommand{\ec}{\end{center}}       % end center
\newcommand\opacity{1}               % opacity 
\pgfsetfillopacity{\opacity}

\newcommand*\Laplace{\mathop{}\!\mathbin\bigtriangleup} % symbole de Laplace

\frenchbsetup{StandardItemLabels=true} % je ne sais plus

\newcommand{\smallO}[1]{\ensuremath{\mathop{}\mathopen{}o\mathopen{}\left(#1\right)}} % petit o

\newcommand{\cit}{\color{blue}\cite} % permet d'avoir les citations de couleur bleues
\newcommand{\bib}{\color{black}\bibitem} % paragraphe biblio en noir et blanc
\newcommand{\bthebiblio}{\color{black} \begin{thebibliography}} % idem necessaire sinon bug a cause de la couleur
\newcommand{\ethebiblio}{\color{black} \end{thebibliography}}   % idem
%%% TIKZ


%% COULEURS 


\definecolor{definitionf}{RGB}{220,252,220}
\definecolor{definitionl}{RGB}{39,123,69}
\definecolor{definitiono}{RGB}{72,148,101}

\definecolor{propositionf}{RGB}{255,216,218}
\definecolor{propositionl}{RGB}{38,38,38}
\definecolor{propositiono}{RGB}{109,109,109}

\definecolor{theof}{RGB}{255,216,218}
\definecolor{theol}{RGB}{160,0,4}
\definecolor{theoo}{RGB}{221,65,100}

\definecolor{avertl}{RGB}{163,92,0}
\definecolor{averto}{RGB}{255,144,0}

\definecolor{histf}{RGB}{241,238,193}

\definecolor{metf}{RGB}{220,230,240}
\definecolor{metl}{RGB}{56,110,165}
\definecolor{meto}{RGB}{109,109,109}


\definecolor{remf}{RGB}{230,240,250}
\definecolor{remo}{RGB}{150,150,150}

\definecolor{exef}{RGB}{240,240,240}

\definecolor{protf}{RGB}{247,228,255}
\definecolor{protl}{RGB}{105,0,203}
\definecolor{proto}{RGB}{174,88,255}

\definecolor{grid}{RGB}{180,180,180}

\definecolor{titref}{RGB}{230,230,230}

\definecolor{vert}{RGB}{23,200,23}

\definecolor{violet}{RGB}{180,0,200}

\definecolor{copper}{RGB}{217, 144, 88}

%% Couleur des ref

\hypersetup{
	colorlinks=true,
	linkcolor=black,
	citecolor=blue,
	urlcolor=black
		   }

%% CADRES

\tikzset{every shadow/.style={opacity=1}}

\global\mdfdefinestyle{doc}{backgroundcolor=white, shadow=true, shadowcolor=propositiono, linewidth=1pt, linecolor=black, shadowsize=5pt}
\global\mdfdefinestyle{titr}{backgroundcolor=metf, shadow=true, shadowcolor=propositiono, linewidth=1pt, linecolor=black, shadowsize=5pt}
\global\mdfdefinestyle{theo}{backgroundcolor=theof, shadow=true, shadowcolor=theoo, linewidth=1pt, linecolor=theol, shadowsize=5pt}
\global\mdfdefinestyle{prop}{backgroundcolor=theof, shadow=true, shadowcolor=propositiono, linewidth=1pt, linecolor=theol, shadowsize=5pt}
\global\mdfdefinestyle{def}{backgroundcolor=definitionf, shadow=true, shadowcolor=definitiono, linewidth=1pt, linecolor=definitionl, shadowsize=5pt}
\global\mdfdefinestyle{histo}{backgroundcolor=histf, shadow=true, shadowcolor=propositiono, linewidth=1pt, linecolor=black, shadowsize=5pt}
\global\mdfdefinestyle{avert}{backgroundcolor=white, shadow=true, shadowcolor=averto, linewidth=1pt, linecolor=avertl, shadowsize=5pt}
\global\mdfdefinestyle{met}{backgroundcolor=metf, shadow=true, shadowcolor=meto, linewidth=1pt, linecolor=metl, shadowsize=5pt}
\global\mdfdefinestyle{rem}{backgroundcolor=metf, shadow=true, shadowcolor=meto, linewidth=1pt, linecolor=metf, shadowsize=5pt}
\global\mdfdefinestyle{exo}{backgroundcolor=exef, shadow=true, shadowcolor=propositiono, linewidth=1pt, linecolor=exef, shadowsize=5pt}
\global\mdfdefinestyle{not}{backgroundcolor=definitionf, shadow=true, shadowcolor=propositiono, linewidth=1pt, linecolor=black, shadowsize=5pt}
\global\mdfdefinestyle{proto}{backgroundcolor=protf, shadow=true, shadowcolor=proto, linewidth=1pt, linecolor=protl, shadowsize=5pt}

%%%%%%
\definecolor{cobalt}{rgb}{0.0, 0.28, 0.67}
\definecolor{applegreen}{rgb}{0.55, 0.71, 0.0}

\usepackage{tcolorbox}
  \tcbuselibrary{most}
  \tcbset{colback=cobalt!5!white,colframe=cobalt!75!black}



\newtcolorbox{definition}[1]{
	colback=applegreen!5!white,
  	colframe=applegreen!65!black,
	fonttitle=\bfseries,
  	title={#1}}
\newtcolorbox{Programme}[1]{
	colback=cobalt!5!white,
  	colframe=cobalt!65!black,
	fonttitle=\bfseries,
  	title={#1}} 
\newtcolorbox{Proposition}[1]{
      colback=theof,%!5!white,
        colframe=theol,%!65!black,
      fonttitle=\bfseries,
        title={#1}}  

\newtcolorbox{Exercice}[1]{
  colback=cobalt!5!white,
  colframe=cobalt!65!black,
  fonttitle=\bfseries,
  title={#1}}  

\newtcolorbox{Resultat}[1]{
	colback=theof,%!5!white,
	colframe=theoo!85!black,
  fonttitle=\bfseries,
	title={#1}} 	

  \setlength{\tabcolsep}{20pt}

  \renewcommand{\arraystretch}{1.5}
  
  \newcommand{\pisteverte}{
	\begin{flushleft}
		\begin{tikzpicture}
			\draw (0,0) -- (0,.2);
			\draw[fill = green] (0,0.4) circle (0.2);
			\node[draw] at (1.5,0.3) {Piste verte};
		\end{tikzpicture}
		\end{flushleft}
}

\newcommand{\pistebleue}{
	\begin{flushleft}
		\begin{tikzpicture}
			\draw (0,0) -- (0,.2);
			\draw[fill = blue] (0,0.4) circle (0.2);
			\node[draw] at (1.5,0.3) {Piste bleue};
		\end{tikzpicture}
		\end{flushleft}
}
\newcommand{\pistenoire}{
	\begin{flushleft}
		\begin{tikzpicture}
			\draw (0,0) -- (0,.2);
			\draw[fill = black!80] (0,0.4) circle (0.2);
			\node[draw] at (1.5,0.3) {Piste noire};
		\end{tikzpicture}
		\end{flushleft}
}
  \newcommand{\titre}[1]{
    \begin{mdframed}[style=titr, leftmargin=0pt, rightmargin=0pt, innertopmargin=8pt, innerbottommargin=8pt, innerrightmargin=10pt, innerleftmargin=10pt]
      \begin{center}
        \Large{\textbf{#1}}
      \end{center}
    \end{mdframed}
  }


  %% COMMANDE Exercice
  
  \newcommand{\exo}[3]{
    \begin{mdframed}[style=exo, leftmargin=0pt, rightmargin=0pt, innertopmargin=8pt, innerbottommargin=8pt, innerrightmargin=10pt, innerleftmargin=10pt]
  
      \noindent \textbf{Exercice #1 - #2}\medskip
  
      #3
    \end{mdframed}
  }
  
     
  \newcommand{\questions}[1]{
    \begin{mdframed}[style=exo, leftmargin=0pt, rightmargin=0pt, innertopmargin=8pt, innerbottommargin=8pt, innerrightmargin=10pt, innerleftmargin=10pt]
  
      \noindent \textbf{Questions :}\smallskip
  
      #1
    \end{mdframed}
  }
  
  \newcommand{\doc}[3]{
    \begin{mdframed}[style=doc, leftmargin=0pt, rightmargin=0pt, innertopmargin=8pt, innerbottommargin=8pt, innerrightmargin=10pt, innerleftmargin=10pt]
  
      \noindent \textbf{Document #1 - #2}\medskip
  
      #3
    \end{mdframed}
  }
\def\width{12}
\def\hauteur{5}


\usetikzlibrary{intersections}
\usetikzlibrary{decorations.markings}
\usetikzlibrary{angles,quotes} % for pic
\usetikzlibrary{calc}
\usetikzlibrary{3d}
\contourlength{1.3pt}

\tikzset{>=latex} % for LaTeX arrow head
\colorlet{myred}{red!85!black}
\colorlet{myblue}{blue!80!black}
\colorlet{mycyan}{cyan!80!black}
\colorlet{mygreen}{green!70!black}
\colorlet{myorange}{orange!90!black!80}
\colorlet{mypurple}{red!50!blue!90!black!80}
\colorlet{mydarkred}{myred!80!black}
\colorlet{mydarkblue}{myblue!80!black}
\tikzstyle{xline}=[myblue,thick]
\def\tick#1#2{\draw[thick] (#1) ++ (#2:0.1) --++ (#2-180:0.2)}
\tikzstyle{myarr}=[myblue!50,-{Latex[length=3,width=2]}]
\def\N{90}

\tikzset{
  % style to apply some styles to each segment of a path
  on each segment/.style={
    decorate,
    decoration={
      show path construction,
      moveto code={},
      lineto code={
        \path [#1]
        (\tikzinputsegmentfirst) -- (\tikzinputsegmentlast);
      },
      curveto code={
        \path [#1] (\tikzinputsegmentfirst)
        .. controls
        (\tikzinputsegmentsupporta) and (\tikzinputsegmentsupportb)
        ..
        (\tikzinputsegmentlast);
      },
      closepath code={
        \path [#1]
        (\tikzinputsegmentfirst) -- (\tikzinputsegmentlast);
      },
    },
  },
  % style to add an arrow in the middle of a path
  mid arrow/.style={postaction={decorate,decoration={
        markings,
        mark=at position .5 with {\arrow[#1]{stealth}}
      }}},
}



\usetikzlibrary{3d, shapes.multipart}

% Styles
\tikzset{>=latex} % for LaTeX arrow head
\tikzset{axis/.style={black, thick,->}}
\tikzset{vector/.style={>=stealth,->}}
\tikzset{every text node part/.style={align=center}}
\usepackage{amsmath} % for \text
 
\usetikzlibrary{decorations.pathreplacing,decorations.markings}

%% MODIFICATION DE CHAPTER  
\makeatletter
\def\@makechapterhead#1{%
  %%%%\vspace*{50\p@}% %%% removed!
  {\parindent \z@ \raggedright \normalfont
    \ifnum \c@secnumdepth >\m@ne
        \huge\bfseries \@chapapp\space \thechapter
        \par\nobreak
        \vskip 20\p@
    \fi
    \interlinepenalty\@M
    \Huge \bfseries #1\par\nobreak
    \vskip 40\p@
  }}
\def\@makeschapterhead#1{%
  %%%%%\vspace*{50\p@}% %%% removed!
  {\parindent \z@ \raggedright
    \normalfont
    \interlinepenalty\@M
    \Huge \bfseries  #1\par\nobreak
    \vskip 40\p@
  }}
  
  \newcommand{\isotope}[3]{%
     \settowidth\@tempdimb{\ensuremath{\scriptstyle#1}}%
     \settowidth\@tempdimc{\ensuremath{\scriptstyle#2}}%
     \ifnum\@tempdimb>\@tempdimc%
         \setlength{\@tempdima}{\@tempdimb}%
     \else%
         \setlength{\@tempdima}{\@tempdimc}%
     \fi%
    \begingroup%
    \ensuremath{^{\makebox[\@tempdima][r]{\ensuremath{\scriptstyle#1}}}_{\makebox[\@tempdima][r]{\ensuremath{\scriptstyle#2}}}\text{#3}}%
    \endgroup%
  }%

\makeatother


\definecolor{darkpastelgreen}{rgb}{0.01, 0.75, 0.24}
\newcommand{\mobiliser}{
  % \begin{flushleft}
    \begin{tikzpicture}[scale=0.6]
      % \draw (0,0) -- (0,.2);
      \draw[color = darkpastelgreen, fill = darkpastelgreen] (0,-0.3) circle (0.3)node[white]{M};
      % \node[draw, white] at (0,-0.3) {\textbf{M}};
    \end{tikzpicture}
    % \end{flushleft}
}

\newcommand{\realiser}{
  % \begin{flushleft}
    \begin{tikzpicture}[scale=.6]
      % \draw (0,0) -- (0,.2);
      \draw[color = blue, fill = blue] (0,-0.3) circle (0.3) node[white]{R};
      % \node[draw, white] at (0,-0.3) {\textbf{R}};
    \end{tikzpicture}
    % \end{flushleft}
}

\definecolor{bostonuniversityred}{rgb}{0.8, 0.0, 0.0}

\newcommand{\analyser}{
  % \begin{flushleft}
    \begin{tikzpicture}[scale=.6]
      % \draw (0,0) -- (0,.2);
      \draw[color = bostonuniversityred, fill = bostonuniversityred] (0,-0.3) circle (0.3) node[white]{A};
      % \node[draw, white] at (0,-0.3) {\textbf{A}};
    \end{tikzpicture}
    % \end{flushleft}
}
\definecolor{amethyst}{rgb}{0.6, 0.4, 0.8}

\newcommand{\communiquer}{
  % \begin{flushleft}
    \begin{tikzpicture}[scale=.6]
      % \draw (0,0) -- (0,.2);
      \draw[color = amethyst, fill = amethyst] (0,-0.3) circle (0.3) node[white]{C};
      % \node[draw, white] at (0,-0.3) {\textbf{C}};
    \end{tikzpicture}
    % \end{flushleft}
}

\newcommand{\applicationnumerique}{\textbf{A.N.:}}

\usepackage{esint}
\usepackage{breqn}
\usepackage{colortbl}
\newcommand{\objectifs}[1]{
	\begin{minipage}{.02\textheight}
	\rotatebox{90}{\textbf{\large Objectifs}}
	\end{minipage}
	\begin{minipage}{.9\linewidth}
			#1 
	\end{minipage}
}
%%
%%
%% DEBUT DU DOCUMENT
%%

\begin{document}

\section*{Leçon 22: Propriétés macroscopiques des corps ferromagnétiques}

\hrulefill\\

\noindent\underline{\textbf{Niveau:}} 
\begin{itemize}
    \item Licence 3
\end{itemize}

\noindent\underline{\textbf{Pré-requis:}}
\begin{itemize}
    \item Électromagnétisme
    \item équations de Maxwell
    \item ARQS
\end{itemize}

\noindent\underline{\textbf{Références:}}

\begin{itemize}
\item Dunod de PSI
\item Garing milieux magnétiques
\item BFR Electromagnetisme 4
\item Perez 
\end{itemize}

\hrulefill

\section*{Introduction}
Il existe des matériaux qualifiés de ferromagnétiques qui conduisent le champ magnétique. Ils permettent de construire de véritables circuits magnétiques dans les transformateurs (permet de faire transiter des puissances électriques sans pertes entre la production et les utilisateurs). 

\section*{1. Aimantation d'un corps ferroagnétique}
Dans la matière, il y a des électrons et des atomes qui portent des moments magnétiques. 


\subsection*{1.1. Les équations de Maxwell dans un milieu matériel}

Rappel des équations de Maxwell dans le vide.
On modifie ces équations en tenant compte de l'éxistence du milieu matériel: les densité volumiques de charge et de courant sont séparées en deux catégories, desdensités liées au milieu matériel et des densités libres  dues au déplacement des porteurs de charge. On peut alors définir deux nouveaux vecteurs: le vecteur polarisation $\vec{\nabla}\cdot \vec{P}=-\rho_{\rm lie}$ et le vecteur aimantation $\vec{\nabla}\wedge \vec{M}=\vec{j}_{\rm lie}$. Les équations Maxwell Flux et Faraday restent inchangées mais les équations liés aux sources s'écrivent:

\begin{equation}
    \vec{\nabla}\cdot \vec{E}=\rho_{lib}-\vec{\nabla}\cdot \vec{P}~\text{et}~\vec{\nabla}\wedge \vec{B}=\mu_0\left(\vec{j}_{lib}+\vec{\nabla}\wedge\vec{M}\right).
\end{equation}

Dans la suite on ne s'intéresse qu'à l'équation de Maxwell-Faraday liant le champ magnétique aux sources de porteurs de charges. On peut définir un nouveau champ, le champ d'excitation magnétique tel que l'équation de Maxwell-Faraday devienne: 

\begin{equation}
    \vec{\nabla}\wedge \vec{H}=\vec{j}_{\rm lib}~\text{où}~\vec{H}=\dfrac{\vec{B}}{\mu_0}-\vec{M}
\end{equation}


\subsection*{1.3. Les différentes formes de magnétisme}

Pour certains corps o; est possible d'écrire une relation entre le champ d'excitation magnétique et l'aimantation:

\begin{equation}
    \vec{M}=\chi_m\vec{H}
\end{equation}

$\chi_m$ est la susceptibilité magnétique, grandeur sans dimension représentant la faculté d'un matériau à s'aimanter. En fonction du signe de $\chi_m$ on définit deux types de matériaux:

\begin{itemize}
    \item les milieux diamagnétiques pour lesquels $\chi_m<0$ avec $|\chi_m|\approx 1\cdot10^{-5}$.
    \item les milieux paramagnétiques pour lesquels $\chi_m>0$ avec $|\chi_m|\approx 1\cdot10^{-3}$.
\end{itemize}

Pour ces milieux, la relation entre $\vec{M}$ et $\vec{H}$ est linéaire. Pour d'autres corps, cette linéarité n'est valable que si le champ magnétique est faible. C'est le cas des ferromagnétiques pour lesquels, lorsque le champ devient plus fort, la relation entre le champ $\vec{H}$ et le champ $\vec{M}$ devient fortement non linéaire jusqu'à saturation.

\subsection*{1.4. Courbe de première aimantation}

On considère initialement un matériau ferromagnétique non aimanté et on augmente progressivement la valeur de l'excitation magnétique. On trace l'évolution du champ magnétique dans le matériau en fonction de l'excitation magnétique. On remarque trois domaines:

\begin{itemize}
    \item Pour une excitation faible la relation entre $\vec{B}$ et $\vec{H}$ est linéaire. Elle correspond à des phénomènes réversibles au sein du milieu.
    \item Le champ magnétique augmente ensuite plus fortement de façon non-linéaire.
\end{itemize}

\textbf{Manipulation:} Transformateur : poly de Philippe sur le magnétisme.

Courbe $M(H)$ pour un matériau ferrmoagnétique initialement non aimanté : (1) courbe linéaire pour les petits $H$ puis (2) fortement croissante puis (3) sature progressivement jusqu'à $M_{sat}$.

$\mu_0 M_{sat}$ est le champ magnétique maximal d'un ferromagnétique à $T$ donnée et dépend du matériau. $M_{sat}(Fe) = 2.1 T$

\subsection*{1.5. Interprétation microscopique (9min12)}
Slide : Domaines de Weiss. Déplacement des domaines réversible à faible $B$, mais irreversible pour $B$ plus élevé du fait de la présence d'impuretés dans le matériau.\\

Une propriété des ferromagnétiques est la canalisation des lignes de champs magnétiques. Slide: illustrations pour différentes géométries. Pour un ferromagnétique torique, les lignes de champ sont complètement canalisées.

\subsection*{1.2. Théorême d'Ampère dans les milieux magnétiques}

On a démontré précédemment que l'on retrouvait le théorême d'Ampère dans le vide à partir de l'équation de Maxwell-Faraday dans le vide. On retrouve par le même moyen une version du théorême d'ampère dans la matière en remplaçant le champ magnétique par le champ d'excitation magnétique: 

\begin{equation}
    \oint_C\vec{H}\cdot d\vec{l}=\sum I_{\rm enlacé}
\end{equation}



\section*{2. Cycle d'hystéresis}

On applique le théorême d'Ampère, on en déduit la relation entre $H$ et $U$ la tension dans le primaire. On écrit l'induction dans le secondaire et on en déduit la tension aux bornes de la résistance. On peut alors tracer B et H . On fait l'experience et on observe l'hysteresis.

\subsection*{2.1. Etude du noyau de fer d'un transformateur(15min)}

Schéma électrique du montage (cf. TP Conversion de puissance électrique)

Théorème d'Ampère : $L H = n_1 i_1 + n_2 i_2$
$L$ : longueur totale du tore (fer doux), $n_i$ nombre de spires de la bobine $i$. $H = \frac{n_1 i_1 + n_2 i_2}{L}$ donne (avec $i_2$ négligeable devant $i_1$ car la résistance imposée dans le secondaire ($\sim 10$~k$\Omega$) associée est bien plus élevée que celle du primaire ($\sim 30$~$\Omega$)) $i_1 = \frac{L}{n_1} H$. Ainsi, si on mesure la tension aux bornes de la résistance du circuit primaire : 

$V_x = R i_1 = \frac{R L}{n_1} H$ avec ici $\frac{R L}{n_1} = 62.4 V/(Am^{-1})$ qu'on place sur la voie 1 de l'oscillo.

Dans le circuit secondaire, on a (Loi de Faraday) : $e = -\frac{d \phi}{d t} = -n_2 S \frac{d B}{d t} = R' i_2 + \int \frac{i_2}{C}$ (cf. TP). $R'$ et $C$ sont choisies de telle sorte que $\int \frac{i_2}{C}$ soit négligeable devant $R' i_2$. Il vient :

$i_2 = \frac{S}{R'} \frac{d B}{d t}$.

Conséquences, aux bornes du condensateur: $U_c = \int \frac{n_2 S}{c R'} \frac{d B}{d t} d t$. Finalement :

$V_y = U_c = \frac{n2 S}{R' C} B$ qu'on place sur la voie 2 de l'oscillo.

\subsection*{2.2. Début de l'expérience}

\begin{itemize}
    \item Visualisation du cycle d'hystérésis avec le mode XY.
    \item Tracé du cycle $B(H)$ au tableau et définition du champ coercitif $H_c$ (pour $B=0$) et du champ rémanent $B_r$ (pour $H=0$).
    \item Mesure expérimentale de $B_r$. Valeurs : $V_y = 2.70 \pm 0.02$~V donne $B_r = 0.532 \pm 0.004 T$. 
    \item Mesure expérimentale de $H_c$. Valeurs : $V_x = 2.10 \pm 0.01$~V donne $H_c = 313 \pm 1$~A/m. Valeur caractéristique des ferro doux. Plus cette valeur est faible, plus l'excitation à devoir appliquer pour désaimanter le matériau sera faible: ce type de matériau se désaimante facilement.  
\end{itemize}

On distingue deux types de ferro (slide tableau comparatif). Ferro doux (Transformateurs, inductance à haute fréquence) et ferro durs (application générateur électrique, RMN, etc.).

\subsection*{2.3. Bilan de puissance (35min38)}

Loi des mailles : $U i_1 + e i_1 - R i_1 = 0$. Premier terme: ; dernier terme : puissance dissipée par effet Joule. $e i_1 = - \frac{d \phi}{d t} i_1$. $\delta W = - d \phi i_1 = - \frac{SHL}{n_1} = d B$. $P = \frac{1}{T} SL \oint H _ud B$. ¨

\section*{Conclusion}

Application : disques durs. Pantographe

\questions{  \textbf{Q: Si je lis votre relation, si j'augmente le nombre de spires, je vais diminuer la perte par hystérésis ?} \textcolor{purple}{Non, il y a une erreur dans ma formule, elle ne dépend pas du nombre de spires.} \newline
  
\textbf{Q: Si je regarde le cycle d'hystérésis, le champ $B$ sature ?} \textcolor{purple}{Non, il continue à croître linéairement.} \newline

 \textbf{Q: Pourquoi on utilise des ferro doux pour l'inductance à haute fréquence.?} \textcolor{purple}{Ce qui compte dans une inductance c'est la variation du flux. Mais dans le ferro dur, quand c'est saturé, certes le champ est fort, mais il n'est plus sensible au champ extérieur. Un ferro doux, en première approximation, c'est linéaire et la pente est la susceptibilité. Mais pour le ferro dur, la pente est nulle.} \newline

\textbf{Q: Pourquoi à haute fréquence ?} \textcolor{purple}{Pour minimiser les pertes par courants de Foucault.} \newline

 \textbf{Q: Quels matériaux qui minimisent ces pertes à hautes fréquence?} \textcolor{purple}{Utiliser des isolants (ferrites), on va limiter ainsi des pertes par courants de Foucault.} \newline

\textbf{Q: Est-ce que la canalisation des champ est générique à tous les ferro ?} \textcolor{purple}{Ce n'est pas le cas pour les ferros durs, que pour les ferros doux. Toutes les applications qui utilisent $\chi$ ou $\mu_r$ très grand c'est les ferro doux, car il n'y a plus de pente pour les ferro durs.} \newline

 \textbf{Q: Dans quel état sont les ferromagnétiques ?} \textcolor{purple}{Solides cristallin. L'état ferro provient d'interaction au niveau des atomes qui n'existent pas à l'état fluide.} \newline

\textbf{Q: L'aimantation à saturation et le champ coercitif dépendent des matériaux ?} \textcolor{purple}{Varie de quelques magnétons de Bohr mais reste du même ordre de grandeur alors que $H_c$ varie beaucoup : un ferro doux a un champ coercitif faible (10$^{-3}$~T), un ferro dur très grand (0.1~T) matériau à un autre.} \newline

 \textbf{Q: Sur l'histoire du transformateur, vous avez appliqué le théorème d'Ampère. C'est évident que $\oint H.d l = H L$ ? } \textcolor{purple}{Il faut utiliser les relations de passage des champs B et H pour démontrer la canalisation des lignes de champs dans le ferro en l'absence de courants surfaciques à l'interface fer$\rightarrow$air. Ensuite, les symétries et invariances du tore donnent $\mathbf{H}=H(r)\mathbf{e_{\theta}}$ et le théorème d'Ampère le long d'une ligne de champ permet d'avoir la formule donnée si on considère que la section du tore est faible devant la distance à l'axe.%Vous n'avez pas utilisé $\nabla \cdot B = 0$, il faut l'utiliser
 .} \newline

\textbf{Q: Dans un éléctroaimant ?} \textcolor{purple}{Il n'y a pas conservation de la norme de $H$. Pour relier $H$ dans l'entre-fer et dans le milieu, il faut utiliser la conservation du flux.} \newline

 \textbf{Q: Pourquoi le système forme les domaines de Weiss ?} \textcolor{purple}{Au niveau microscopique, il y a une compétition entre l'énergie qu'il va falloir fournir pour créer ces interfaces et le coût en énergie pour créer un champ via l'alignement des moments magnétiques.} \newline

\textbf{Q: La taille des domaines ?} \textcolor{purple}{De l'ordre du micromètre.} \newline

 \textbf{Q: Est-ce qu'il y a des directions priviligiées au départ dans champ extérieur? Est-ce que je peux avoir une courbe d'hystérésis qui dépend du champ $B$?} \textcolor{purple}{Oui, il y a un axe de facile aimantation. Les champs coercitifs vont être plus forts dans l'axe de facile aimantation. Cela va exister dans des monocristaux. Les axes de facile et difficile aimantation sont définis par rapport à l'orientation cristalline du matériau et des interactions entre les moments magnétiques dans le matériau. Il n'y a pas de raison pour que l'orientation du domaine soit dans la direction du champ appliqué.} \newline

\textbf{Q:  Quels sont les conditions qui vous permettent de lire directement $B$?} \textcolor{purple}{$e = U_{R'} + U_c = R' i_2 + \frac{1}{jC\omega} i_2 = R'(1+\frac{1}{jRC\omega}) i_2$. On veut alors $R'C >> \frac{1}{\omega}$ avec $\omega=2\pi f$ et $f=50$~Hz (fréquence du secteur).} \newline

\textbf{Q: Les applications des ferros durs?} \textcolor{purple}{Tous les aimants permanents. } 
}

\end{document}

%%
%% FIN DU DOCUMENT
%%
