%!TEX encoding = UTF-8 Unicode
\documentclass[french, a4paper, 10pt, twocolumn, landscape]{article}



%% Langue et compilation

\usepackage[utf8]{inputenc}
\usepackage[T1]{fontenc}
\usepackage[french]{babel}
\usepackage{lmodern}       % permet d'avoir certains "fonts" de bonne qualite
\renewcommand{\familydefault}{\sfdefault}
%% LISTE DES PACKAGES

\usepackage{mathtools}     % package de base pour les maths
\usepackage{amsmath}       % mathematical type-setting
\usepackage{amssymb}       % symbols speciaux pour les maths
\usepackage{textcomp}      % symboles speciaux pour el text
\usepackage{gensymb}       % commandes generiques \degree etc...
\usepackage{tikz}          % package graphique
\usepackage{wrapfig}       % pour entourer a cote d'une figure
\usepackage{color}         % package des couleurs
\usepackage{xcolor}        % autre package pour les couleurs
\usepackage{pgfplots}      % pacakge pour creer des graph
\usepackage{epsfig}        % permet d'inclure des graph en .eps
\usepackage{graphicx}      % arguments dans includegraphics
\usepackage{pdfpages}      % permet d'insérer des pages pdf dans le document
\usepackage{subfig}        % permet de creer des sous-figure
% \usepackage{pst-all}       % utile pour certaines figures en pstricks
\usepackage{lipsum}        % package qui permet de faire des essais
\usepackage{array}         % permet de faire des tableaux
\usepackage{multicol}      % plusieurs colonnes sur une page
\usepackage{enumitem}      % pro­vides user con­trol: enumerate, itemize and description
\usepackage{hyperref}      % permet de creer des hyperliens dans le document
\usepackage{lscape}        % permet de mettre une page en mode paysage

\usepackage{fancyhdr}      % Permet de mettre des informations en hau et en bas de page      
\usepackage[framemethod=tikz]{mdframed} % breakable frames and coloured boxes
\usepackage[top=1.8cm, bottom=1.8cm, left=1.5cm, right=1.5cm]{geometry} % donne les marges
\usepackage[font=normalsize, labelfont=bf,labelsep=endash, figurename=Figure]{caption} % permet de changer les legendes des figures
\setlength{\parskip}{0pt}%
\setlength{\parindent}{18pt}
\usepackage{lewis}
\usepackage{bohr}
\usepackage{chemfig}
\usepackage{chemist}
\usepackage{tabularx}
\usepackage{pgf-spectra} % permet de tracer des spectres lumineux des atomes et des ions
\usepackage{pgf}

\usepackage{flexisym}
\usepackage{soul}
\usepackage{ulem}
\usepackage{cancel}

\usepackage{import}
\usepackage{physics}
\usepackage[outline]{contour} % glow around text
\tikzset{every shadow/.style={opacity=1}}


%% LIBRAIRIES

\usetikzlibrary{plotmarks} % librairie pour les graphes
\usetikzlibrary{patterns}  % necessaire pour certaines choses predefinies sur tikz
\usetikzlibrary{shadows}   % ombres des encadres
\usetikzlibrary{backgrounds} % arriere plan des encadres


%% MISE EN PAGE

\pagestyle{fancy}     % Défini le style de la page

\renewcommand{\headrulewidth}{0pt}      % largeur du trait en haut de la page
\fancyhead[L]{\textbf{\textcolor{cyan}{Cours}} - Thème 4 - La Terre un astre singulier}         % info coin haut gauche
\fancyhead[R]{\textit{Première Enseignement Scientifique}}  % info coin haut droit

% % bas de la page
% \renewcommand{\footrulewidth}{0pt}      % largeur du trait en bas de la page
% \fancyfoot[L]{}  % info coin bas gauche
\fancyfoot[R]{Lycée GT Jean Guéhenno}                         % info coin bas droit


\setlength{\columnseprule}{1pt} 
\setlength{\columnsep}{30pt}



%% NOUVELLES COMMANDES 

\DeclareMathOperator{\e}{e} % permet d'ecrire l'exponentielle usuellement


\newcommand{\gap}{\vspace{0.15cm}}   % defini une commande pour sauter des lignes
\renewcommand{\vec}{\overrightarrow} % permet d'avoir une fleche qui recouvre tout le vecteur
\newcommand{\bi}{\begin{itemize}}    % begin itemize
\newcommand{\ei}{\end{itemize}}      % end itemize
\newcommand{\bc}{\begin{center}}     % begin center
\newcommand{\ec}{\end{center}}       % end center
\newcommand\opacity{1}               % opacity 
\pgfsetfillopacity{\opacity}

\newcommand*\Laplace{\mathop{}\!\mathbin\bigtriangleup} % symbole de Laplace

\frenchbsetup{StandardItemLabels=true} % je ne sais plus

\newcommand{\smallO}[1]{\ensuremath{\mathop{}\mathopen{}o\mathopen{}\left(#1\right)}} % petit o

\newcommand{\cit}{\color{blue}\cite} % permet d'avoir les citations de couleur bleues
\newcommand{\bib}{\color{black}\bibitem} % paragraphe biblio en noir et blanc
\newcommand{\bthebiblio}{\color{black} \begin{thebibliography}} % idem necessaire sinon bug a cause de la couleur
\newcommand{\ethebiblio}{\color{black} \end{thebibliography}}   % idem
%%% TIKZ


%% COULEURS 


\definecolor{definitionf}{RGB}{220,252,220}
\definecolor{definitionl}{RGB}{39,123,69}
\definecolor{definitiono}{RGB}{72,148,101}

\definecolor{propositionf}{RGB}{255,216,218}
\definecolor{propositionl}{RGB}{38,38,38}
\definecolor{propositiono}{RGB}{109,109,109}

\definecolor{theof}{RGB}{255,216,218}
\definecolor{theol}{RGB}{160,0,4}
\definecolor{theoo}{RGB}{221,65,100}

\definecolor{avertl}{RGB}{163,92,0}
\definecolor{averto}{RGB}{255,144,0}

\definecolor{histf}{RGB}{241,238,193}

\definecolor{metf}{RGB}{220,230,240}
\definecolor{metl}{RGB}{56,110,165}
\definecolor{meto}{RGB}{109,109,109}


\definecolor{remf}{RGB}{230,240,250}
\definecolor{remo}{RGB}{150,150,150}

\definecolor{exef}{RGB}{240,240,240}

\definecolor{protf}{RGB}{247,228,255}
\definecolor{protl}{RGB}{105,0,203}
\definecolor{proto}{RGB}{174,88,255}

\definecolor{grid}{RGB}{180,180,180}

\definecolor{titref}{RGB}{230,230,230}

\definecolor{vert}{RGB}{23,200,23}

\definecolor{violet}{RGB}{180,0,200}

\definecolor{copper}{RGB}{217, 144, 88}

%% Couleur des ref

\hypersetup{
	colorlinks=true,
	linkcolor=black,
	citecolor=blue,
	urlcolor=black
		   }

%% CADRES

\tikzset{every shadow/.style={opacity=1}}

\global\mdfdefinestyle{doc}{backgroundcolor=white, shadow=true, shadowcolor=propositiono, linewidth=1pt, linecolor=black, shadowsize=5pt}
\global\mdfdefinestyle{titr}{backgroundcolor=metf, shadow=true, shadowcolor=propositiono, linewidth=1pt, linecolor=black, shadowsize=5pt}
\global\mdfdefinestyle{theo}{backgroundcolor=theof, shadow=true, shadowcolor=theoo, linewidth=1pt, linecolor=theol, shadowsize=5pt}
\global\mdfdefinestyle{prop}{backgroundcolor=theof, shadow=true, shadowcolor=propositiono, linewidth=1pt, linecolor=theol, shadowsize=5pt}
\global\mdfdefinestyle{def}{backgroundcolor=definitionf, shadow=true, shadowcolor=definitiono, linewidth=1pt, linecolor=definitionl, shadowsize=5pt}
\global\mdfdefinestyle{histo}{backgroundcolor=histf, shadow=true, shadowcolor=propositiono, linewidth=1pt, linecolor=black, shadowsize=5pt}
\global\mdfdefinestyle{avert}{backgroundcolor=white, shadow=true, shadowcolor=averto, linewidth=1pt, linecolor=avertl, shadowsize=5pt}
\global\mdfdefinestyle{met}{backgroundcolor=metf, shadow=true, shadowcolor=meto, linewidth=1pt, linecolor=metl, shadowsize=5pt}
\global\mdfdefinestyle{rem}{backgroundcolor=metf, shadow=true, shadowcolor=meto, linewidth=1pt, linecolor=metf, shadowsize=5pt}
\global\mdfdefinestyle{exo}{backgroundcolor=exef, shadow=true, shadowcolor=propositiono, linewidth=1pt, linecolor=exef, shadowsize=5pt}
\global\mdfdefinestyle{not}{backgroundcolor=definitionf, shadow=true, shadowcolor=propositiono, linewidth=1pt, linecolor=black, shadowsize=5pt}
\global\mdfdefinestyle{proto}{backgroundcolor=protf, shadow=true, shadowcolor=proto, linewidth=1pt, linecolor=protl, shadowsize=5pt}

%%%%%%
\definecolor{cobalt}{rgb}{0.0, 0.28, 0.67}
\definecolor{applegreen}{rgb}{0.55, 0.71, 0.0}

\usepackage{tcolorbox}
  \tcbuselibrary{most}
  \tcbset{colback=cobalt!5!white,colframe=cobalt!75!black}



\newtcolorbox{definition}[1]{
	colback=applegreen!5!white,
  	colframe=applegreen!65!black,
	fonttitle=\bfseries,
  	title={#1}}
\newtcolorbox{Programme}[1]{
	colback=cobalt!5!white,
  	colframe=cobalt!65!black,
	fonttitle=\bfseries,
  	title={#1}} 
\newtcolorbox{Proposition}[1]{
      colback=theof,%!5!white,
        colframe=theol,%!65!black,
      fonttitle=\bfseries,
        title={#1}}  

\newtcolorbox{Exercice}[1]{
  colback=cobalt!5!white,
  colframe=cobalt!65!black,
  fonttitle=\bfseries,
  title={#1}}  

\newtcolorbox{Resultat}[1]{
	colback=theof,%!5!white,
	colframe=theoo!85!black,
  fonttitle=\bfseries,
	title={#1}} 	

  \setlength{\tabcolsep}{20pt}

  \renewcommand{\arraystretch}{1.5}
  
  \newcommand{\pisteverte}{
	\begin{flushleft}
		\begin{tikzpicture}
			\draw (0,0) -- (0,.2);
			\draw[fill = green] (0,0.4) circle (0.2);
			\node[draw] at (1.5,0.3) {Piste verte};
		\end{tikzpicture}
		\end{flushleft}
}

\newcommand{\pistebleue}{
	\begin{flushleft}
		\begin{tikzpicture}
			\draw (0,0) -- (0,.2);
			\draw[fill = blue] (0,0.4) circle (0.2);
			\node[draw] at (1.5,0.3) {Piste bleue};
		\end{tikzpicture}
		\end{flushleft}
}
\newcommand{\pistenoire}{
	\begin{flushleft}
		\begin{tikzpicture}
			\draw (0,0) -- (0,.2);
			\draw[fill = black!80] (0,0.4) circle (0.2);
			\node[draw] at (1.5,0.3) {Piste noire};
		\end{tikzpicture}
		\end{flushleft}
}
  \newcommand{\titre}[1]{
    \begin{mdframed}[style=titr, leftmargin=0pt, rightmargin=0pt, innertopmargin=8pt, innerbottommargin=8pt, innerrightmargin=10pt, innerleftmargin=10pt]
      \begin{center}
        \Large{\textbf{#1}}
      \end{center}
    \end{mdframed}
  }


  %% COMMANDE Exercice
  
  \newcommand{\exo}[3]{
    \begin{mdframed}[style=exo, leftmargin=0pt, rightmargin=0pt, innertopmargin=8pt, innerbottommargin=8pt, innerrightmargin=10pt, innerleftmargin=10pt]
  
      \noindent \textbf{Exercice #1 - #2}\medskip
  
      #3
    \end{mdframed}
  }
  
     
  \newcommand{\questions}[1]{
    \begin{mdframed}[style=exo, leftmargin=0pt, rightmargin=0pt, innertopmargin=8pt, innerbottommargin=8pt, innerrightmargin=10pt, innerleftmargin=10pt]
  
      \noindent \textbf{Questions :}\smallskip
  
      #1
    \end{mdframed}
  }
  
  \newcommand{\doc}[3]{
    \begin{mdframed}[style=doc, leftmargin=0pt, rightmargin=0pt, innertopmargin=8pt, innerbottommargin=8pt, innerrightmargin=10pt, innerleftmargin=10pt]
  
      \noindent \textbf{Document #1 - #2}\medskip
  
      #3
    \end{mdframed}
  }
\def\width{12}
\def\hauteur{5}


\usetikzlibrary{intersections}
\usetikzlibrary{decorations.markings}
\usetikzlibrary{angles,quotes} % for pic
\usetikzlibrary{calc}
\usetikzlibrary{3d}
\contourlength{1.3pt}

\tikzset{>=latex} % for LaTeX arrow head
\colorlet{myred}{red!85!black}
\colorlet{myblue}{blue!80!black}
\colorlet{mycyan}{cyan!80!black}
\colorlet{mygreen}{green!70!black}
\colorlet{myorange}{orange!90!black!80}
\colorlet{mypurple}{red!50!blue!90!black!80}
\colorlet{mydarkred}{myred!80!black}
\colorlet{mydarkblue}{myblue!80!black}
\tikzstyle{xline}=[myblue,thick]
\def\tick#1#2{\draw[thick] (#1) ++ (#2:0.1) --++ (#2-180:0.2)}
\tikzstyle{myarr}=[myblue!50,-{Latex[length=3,width=2]}]
\def\N{90}

\tikzset{
  % style to apply some styles to each segment of a path
  on each segment/.style={
    decorate,
    decoration={
      show path construction,
      moveto code={},
      lineto code={
        \path [#1]
        (\tikzinputsegmentfirst) -- (\tikzinputsegmentlast);
      },
      curveto code={
        \path [#1] (\tikzinputsegmentfirst)
        .. controls
        (\tikzinputsegmentsupporta) and (\tikzinputsegmentsupportb)
        ..
        (\tikzinputsegmentlast);
      },
      closepath code={
        \path [#1]
        (\tikzinputsegmentfirst) -- (\tikzinputsegmentlast);
      },
    },
  },
  % style to add an arrow in the middle of a path
  mid arrow/.style={postaction={decorate,decoration={
        markings,
        mark=at position .5 with {\arrow[#1]{stealth}}
      }}},
}



\usetikzlibrary{3d, shapes.multipart}

% Styles
\tikzset{>=latex} % for LaTeX arrow head
\tikzset{axis/.style={black, thick,->}}
\tikzset{vector/.style={>=stealth,->}}
\tikzset{every text node part/.style={align=center}}
\usepackage{amsmath} % for \text
 
\usetikzlibrary{decorations.pathreplacing,decorations.markings}

%% MODIFICATION DE CHAPTER  
\makeatletter
\def\@makechapterhead#1{%
  %%%%\vspace*{50\p@}% %%% removed!
  {\parindent \z@ \raggedright \normalfont
    \ifnum \c@secnumdepth >\m@ne
        \huge\bfseries \@chapapp\space \thechapter
        \par\nobreak
        \vskip 20\p@
    \fi
    \interlinepenalty\@M
    \Huge \bfseries #1\par\nobreak
    \vskip 40\p@
  }}
\def\@makeschapterhead#1{%
  %%%%%\vspace*{50\p@}% %%% removed!
  {\parindent \z@ \raggedright
    \normalfont
    \interlinepenalty\@M
    \Huge \bfseries  #1\par\nobreak
    \vskip 40\p@
  }}
  
  \newcommand{\isotope}[3]{%
     \settowidth\@tempdimb{\ensuremath{\scriptstyle#1}}%
     \settowidth\@tempdimc{\ensuremath{\scriptstyle#2}}%
     \ifnum\@tempdimb>\@tempdimc%
         \setlength{\@tempdima}{\@tempdimb}%
     \else%
         \setlength{\@tempdima}{\@tempdimc}%
     \fi%
    \begingroup%
    \ensuremath{^{\makebox[\@tempdima][r]{\ensuremath{\scriptstyle#1}}}_{\makebox[\@tempdima][r]{\ensuremath{\scriptstyle#2}}}\text{#3}}%
    \endgroup%
  }%

\makeatother


\definecolor{darkpastelgreen}{rgb}{0.01, 0.75, 0.24}
\newcommand{\mobiliser}{
  % \begin{flushleft}
    \begin{tikzpicture}[scale=0.6]
      % \draw (0,0) -- (0,.2);
      \draw[color = darkpastelgreen, fill = darkpastelgreen] (0,-0.3) circle (0.3)node[white]{M};
      % \node[draw, white] at (0,-0.3) {\textbf{M}};
    \end{tikzpicture}
    % \end{flushleft}
}

\newcommand{\realiser}{
  % \begin{flushleft}
    \begin{tikzpicture}[scale=.6]
      % \draw (0,0) -- (0,.2);
      \draw[color = blue, fill = blue] (0,-0.3) circle (0.3) node[white]{R};
      % \node[draw, white] at (0,-0.3) {\textbf{R}};
    \end{tikzpicture}
    % \end{flushleft}
}

\definecolor{bostonuniversityred}{rgb}{0.8, 0.0, 0.0}

\newcommand{\analyser}{
  % \begin{flushleft}
    \begin{tikzpicture}[scale=.6]
      % \draw (0,0) -- (0,.2);
      \draw[color = bostonuniversityred, fill = bostonuniversityred] (0,-0.3) circle (0.3) node[white]{A};
      % \node[draw, white] at (0,-0.3) {\textbf{A}};
    \end{tikzpicture}
    % \end{flushleft}
}
\definecolor{amethyst}{rgb}{0.6, 0.4, 0.8}

\newcommand{\communiquer}{
  % \begin{flushleft}
    \begin{tikzpicture}[scale=.6]
      % \draw (0,0) -- (0,.2);
      \draw[color = amethyst, fill = amethyst] (0,-0.3) circle (0.3) node[white]{C};
      % \node[draw, white] at (0,-0.3) {\textbf{C}};
    \end{tikzpicture}
    % \end{flushleft}
}

\newcommand{\applicationnumerique}{\textbf{A.N.:}}

\usepackage{esint}
\usepackage{breqn}
\usepackage{colortbl}
\newcommand{\objectifs}[1]{
	\begin{minipage}{.02\textheight}
	\rotatebox{90}{\textbf{\large Objectifs}}
	\end{minipage}
	\begin{minipage}{.9\linewidth}
			#1 
	\end{minipage}
}
%%
%%
%% DEBUT DU DOCUMENT
%%

\begin{document}
\section*{Leçon 27: Effet tunnel, application à la réactivité alpha}

\hrulefill\\

\noindent\underline{\textbf{Niveau:}} 
\begin{itemize}
    \item Deuxième année CPGE
\end{itemize}

\noindent\underline{\textbf{Pré-requis:}}
\begin{itemize}
    \item Physique ondulatoire
    \item Notion de mécanique quantique (eq de Schrödinger stationnaire, puits de potentiels)
    \item radioactivité
\end{itemize}

\noindent\underline{\textbf{Références:}}

\begin{itemize}
    \item Dunoc PC 
    \item Berkley quantique
    \item 51 leçons pour l'agreg
    \item sujet des mines
\end{itemize}

\hrulefill


\section*{Introduction}

Avec le puit de potentiel et la barrière de potentiel, on a vu que la fonction d'onde pouvait déborder sur des zones ou le potentiel est plus grand que son énergie. On peut le montrer via un programme python, par exemple avec la marche que l'on peut trouver en exercice (exo 34.6 Dunod PC 2019 par exemple). Ou comme dans le dunod avec un puit de potentiel carré.\medskip

On a vu dans le cours précédent qu'une particule quantique dans un puit de potentiel fini présente des niveaux d'énergie qui sont quantifiés:
\begin{equation}
    E_n = n^2\dfrac{k^2\hbar^2}{2m}
\end{equation}

où $n$ est le $n-ème$ état quantique de la particule tel que $n=1$ est le fondamental. $a$ est la largeur du puit de potentiel. $\hbar$ la constante de Planck et $m$ la masse de la particule. La recherche des fonctions d'ondes propres vérifiant Schrödinger stationnaire conduit à la recherche de mode propres de vibration (un peu comme la corde de Melde).Utiliser l'analogie de la corde de Melde que les étudiants connaissent.  Dans les deux cas l'écriture des conditions aux limites impose une quantification des vecteurs d'ondes  et revient à écrire $a=n\lambda_n/2$.  On présente alors le programme python Du puit carré. On peut présenter les trois premiers états par exemple.\medskip

On pourrait sortir une corde de Melde pour le montrer qualitativement mais ce n'est pas l'objet de la leçon et pas le temps. Noeuds = densité de probabilité nulle, ventres = densité de proba maximale.

Pour une particule classique, les zones pour $|x|>a/2$ sont interdites. En revanche on remarque que pour une particule quantique, la probabilité d'obtenir une particule au-delà de la barrière est non nulle ! L'expression de la fonction d'onde dans les zones I et III (zone interdite) sont de la forme :

\begin{equation}
    \psi(x) = A{\rm e}^{\pm qx}
\end{equation}

C'est l'expression d'une onde évanescente, dont on peut définir une profondeur de pénétration de la particule quantique d'énergie $E<V_0$ dans les régions interdites par la mécanique classique:

\begin{equation}
    \delta = \dfrac{\hbar}{\sqrt{2m(V_0-E)}}
\end{equation}

Comment peut-on utiliser cette propriété ? 

\section*{1. Barrière de potentiel}

\subsection*{1.1. Fonction d'onde prore}
On suit le Dunod PC p $1261$, on résoud Schrödinger stationnaire dans les trois domaines donnés.


Probabilité : $|\psi(x,t)|^2dx$ = probabilité de trouver la particule entre x et x+dx. Interprétation possible que si $\int_{-\infty}^{+\infty}|\psi(x,t)|^2dx=1$.\\

Equation de Schrodinger : $i\hbar\frac{\partial \psi(x,t)}{\partial t}=-\frac{\hbar^2}{2m}\frac{\partial^2\psi(x,t)}{\partial x^2}+V(x)\psi(x,t)$\\

Equation linéaire, on peut choisir $\psi = \chi(x)\exp\left(-\frac{iE}{h}t\right)$ avec le signe - dans l'exponentiel pour faire correspondre l'énergie de la particule.\\

Energie = valeur propre de H, c'est ce qu'on cherche à résoudre tout le temps en mécanique quantique. 
\begin{itemize}
    \item Si V confinement $V\sim x^2$, alors E est quantifié. on a des \textbf{états liés}
    \item Si V non confinement, E est continue et on a des \textbf{états de diffusion (ou libres)}
\end{itemize}

Etats de diffusion : ce sont les états qui vont nous intéresser pour l'effet tunnel. Comment les interpréter ?


\subsection*{1.2. Probabilité de réflexion et de transmission. Effet tunnel}

Toujours Dunod PC. On donne les fonctions d'ondes incidentes et réfléchies ainsi que les vecteurs densités de proba. Pour arriver a définir les coefficients de réflexion et transmission.

\begin{equation}
    \begin{array}{lll}
        R = \dfrac{||\vec{j}_r||}{||\vec{j}_i||}& \text{ et }& T=\dfrac{||\vec{j}_t||}{||\vec{j}_i||}
    \end{array}
\end{equation}

On écrit les conditions aux limites pour déterminer les coefficients. On en déduit $R$ et $T$.

\subsection*{1.3. Approximation d'une barrière épaisse}

On s'intéresse au cas d'une barrière épaisse, c'est à dire que sa largeur a est très grande devant la profondeur de pénétratin $a\ll \delta$. Cela correspond à $a$ grand ou $V_0 \ll E$. (particules de faible énergie devant la hauteur de la barrière).   On peut alors simplifier l'expression de $T$ car $\sinh(qa)\sim {\rm e}^{2qa}/4\ll 1$. Le coefficient de transmission devient : 

\begin{equation}
    T\approx \dfrac{16E(V_0-E)}{V_0^2}{\rm e}^{-2qa}
\end{equation}

On peut donner en transparents différents ordres de grandeurs pour différentes particules. On choisit d'étudier la radioactivité $\alpha$.

\section*{2. Radioactivité $\alpha$}



\subsection*{2.1. Description et résultats expérimentaux}
Phénomène naturel qui résulte de l 'instabilité d'un noyau atomique qui se désintègre (DUNOD PC 2019 p 1271). Rappeler que c'est l'emission d'une particule $\alpha$ par un noyau instable. Il s'agit d'un noyau d'Helium très stable (2 protons et 2 Neutrons). On peut donner en transparents des exemples de désintégration (Uranium ou radium par exemple).

On peut montrer les résultats expérimentaux \url{https://link.springer.com/article/10.1140/epja/i2019-12804-5}

On constate que le temps de demi-vie est d'autant plus court que l'éergie cinétique de la particule est grande on expliquera ce temps de demi-vie grâce à l'effet tunnel.

\subsection*{2.2. Théorie de la radioactivité $\alpha$: Gamoz Gurney et Condon (BUP 734, MINES PC 2016)}


La radioactivité $\alpha $ a été interprété en 1928 par Gamow grâce à l'effet tunnel. Il considère que le noyau X était constitué au préalable de la particule $\alpha$ et du noyau Y. L'énergie potentielle $V(x)$ (interaction forte de courte portée et de la répulsion électrostatique) entre les deux particules est une fonction de la distance qui les sépare. À l'extérieur c'est le potentiel de Coulomb qui s'applique à la particule. Dessiner le profil du potentiel.\medskip

On mène les calculs comme sur le corrigé des mines. On remarque que la particule $\alpha$ devrait rester piégée dans le puit. Elle s'explique par l'existence d'un effet tunnel: la particule doit travers la barrière par effet tunnel sur une distance. On calcule cette distance ppour l'énergie indiquée dans le tableau de l'article.

On arrive ensuite à l'expression de $\ln{T}$

\subsection*{2.3. Détermination du temps de vie}

On connaît l'expression du coefficient de transmission T. On approxime la barrière variant continument par plusieurs barrières rectangulaires. Le coefficient de transmission global  est le produit des coefficients de transmission. 

\begin{equation}
    \ln(T)=a-\dfrac{b}{\sqrt{E}}
\end{equation}

On constate que T décroit lorsque E augmente ou lorsque la masse decroit. La particule fait des allers-retours dans le noyau et ne cesse de rebondir contre la barrière de potentiel. À chaque collision elle a une proba d'être transmise T. 

Entre deux rebonds sur la barrière, la particule parcours la distance $4x_0$. Donc $t_m = 4x_0/v$. Le nombre moyen de rebonds par seconde:

$$N=1/t_m$$

et la probabilité d'émission $\alpha$ s'écrit $$dp=NTdt = \dfrac{T}{t_m}dt$$

Soit la variation du nombre de noyaux $$dN=-Ndp=-N\dfrac{T}{t_m}dt$$

Soit $$N(t) = N_0{\rm e}^{-Tt/tm}$$

La demi-vie est définie comme le temps après lequel la moitié des noyaux s'est désintégré: $$N(\tau_{1/2})=\dfrac{N_0}{2}$$
Donc $$\tau_{1/2}=t_m\ln{2}/T$$

 $$\ln{\tau_{1/2}}=\ln\ln 2+\ln t_m-\ln T = \underbrace{\ln\ln 2+\ln t_m-a}_{\rm constante}+\dfrac{b}{\sqrt{E}}$$

 On retrouve la loi expérimentale !  Peut-on tracer les données à partir du tableau ? 

 \section*{Conclusion}

 Effet tunnel bien utile pour comprendre des phénomènes naturel. On en a pas parlé mais il est également utilisé pour des application technologiques actuelles : microscope à effet tunnel \url{https://toutestquantique.fr/tunnel/}.


\end{document}

%%
%% FIN DU DOCUMENT
%%
